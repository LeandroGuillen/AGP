\documentclass[a4paper,10pt]{article}
\usepackage[utf8x]{inputenc}


%opening
\title{Aplicación de la Geometría Proyectiva en Criptografía}
\author{Leandro J. Guillén Moreno}

\begin{document}

\maketitle

\begin{abstract}
Historia de las curvas elípticas y la geometría proyectiva. Operaciones con puntos en curvas elípticas. Aplicación en la criptografía de las curvas elípticas.

\emph{History of elliptic curves and projective geometry. Operating with points in elliptic curves. Elliptic curves applied to criptography.}
\end{abstract}

\section{Introducción}
Una curva elíptica es una curva de la forma: $$y^{2}=x^{3}+ax+b$$
Hola \cite{lamport94}.


\section{Correo de Leandro}
Correo recibido del profesor Leandro Marín

\begin{quotation}
Hola,
Precisamente a ti te veía bastante bien en el tema. Yo os daba libertad pensando en que fuera mejor para vosotros, pero no debéis complicaros la vida con el trabajo. Si te parece bien, hazme una explicación de cómo funcionan las operaciones de puntos en curvas elípticas con todos los casos posibles (tangentes, punto del infinito, etc) con sus respectivos gráficos y luego completalo con algo que encuentres de la historia de las curvas elípticas y cómo se aplicaron a la criptografía. No es necesario mucho más y las 10 páginas las sacas enseguida.

Es una sugerencia, si no te gusta, buscamos algo diferente.

Leandro
\end{quotation}

\bibliographystyle{plain}
\bibliography{bibliografia}
\end{document}
